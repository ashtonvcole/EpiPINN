\documentclass{article}
\usepackage{graphicx, float} % Required for inserting images
\usepackage[letterpaper, margin=1in]{geometry}
\usepackage{amsfonts, amsmath, hyperref, listings, multirow, physics, placeins, thmtools}

\DeclareMathOperator{\Grad}{grad}
\DeclareMathOperator{\Div}{div}
\DeclareMathOperator{\Curl}{curl}

\newcommand{\vddot}{\boldsymbol{:}}

\title{EpiPINN: Neural Network for Parameter Estimation of Epidemic Model}
\author{Ashton Cole}
\date{\today}

\begin{document}
	\maketitle
	
	\section{Introduction}
	\label{section:Introduction}
	
	In this project, we attempted to replicate the work of \cite{}, using a physics-informed neural network (PINN) to learn epidemic data and estimate the parameters for a fractional-order SEIRD model. We had some difficulties in implementing parts of the paper, and the PINN did not learn SEIRD parameters accurately. In conclusion, more development is necessary for PINNs to be a reliable choice for epidemic modeling.
	
	\subsection{Premise of Paper}
	\label{subsection:Premise_of_Paper}
	
	\subsection{Caputo Fractional Calculus}
	\label{subsection:Caputo_Fractional_Calculus}
	
	\subsection{SEIRD Model}
	\label{subsection:SEIRD_Model}
	
	note limitation of fractional seird
	
	\section{Methodology}
	\label{section:Methodology}
	
	\subsection{Implementation of Caputo Fractional Calculus}
	\label{subsection:Implementation_of_Caputo_Fractional_Calculus}
	
	\subsection{Implementation of Neural Network}
	\label{subsection:Implementation_of_Neural_Network}
	
	\subsection{Training Process}
	\label{subsection:Training_Process}
	
	topic par
	explain our similar approach, why we did what we did
	data generation
	implementation of fractional calculus
	structure of nn, design choices, logit
	collocation points
	loss function
	training process
	
	\section{Results and Discussion}
	\label{section:Results_and_Discussion}
	
	topic par
	results
	
	\section{Conclusions}
	\label{section:Conclusions}
	
	conclusion paragraph
\end{document}